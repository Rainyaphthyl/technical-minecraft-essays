\documentclass[a4paper, UTF8, 11pt, twoside]{ctexart}
\usepackage{amsmath, amsthm, amssymb, graphicx}
\usepackage[bookmarks=true, colorlinks, citecolor=blue, linkcolor=black]{hyperref}
\usepackage[left=2.54cm, right=2.54cm, top=3.18cm, bottom=3.18cm]{geometry}

% 导言区

\title{刷怪概率与速率的计算}
\author{Naftalluvia}

\begin{document}

\maketitle

\begin{center}
    \LaTeX
\end{center}

\section{引言}

\subsection{游戏版本声明}

在没有特殊说明的情况下,本文所有讨论内容均默认只考虑MCJE 1.12.2的情况,没有验证对于其他版本的适用性。即使是提到了版本差异的段落,对于其他版本的讨论内容也是未经验证的。

本文的内容对其他版本仍然有参考意义,但引用时需要仔细甄别其中的各种版本差异。

\section{单方块的刷怪效率计算}

\subsection{参数设置}

考虑以下与刷怪过程有关的随机变量:

\begin{itemize}
    \item $\vec{S}$ :生物最终生成的位置;
    \item $\vec{A}_k$ :剩余 $k$ 次游走机会时,游走选中的位置;
    \item $\vec{I}$ :一组刷怪尝试中,游走的起点坐标;
    \item $B$ :一组刷怪尝试中,游走的次数;
    \item $\xi$ :一组刷怪尝试中,生物生成列表条目(`SpawnListEntry`),包含生物种类、成群生成数量范围、选择权重。
\end{itemize}

其中,
\begin{itemize}
    \item $\vec{S}, \vec{A}_k, \vec{I}$ 均为二元向量,表示生物生成位置的 $\left(X, Z\right)$ 方块坐标;
    \item $B$ 的取值为正整数,具体范围与 $\xi$ 的取值有关;
    \item $\vec{S}$ 可以视为 $\vec{A}_k$ 的特殊情况: $\vec{S} = \vec{A}_0$ ,即最后一次游走选中的位置即为实际生成位置;
    \item $\vec{I}$ 并不能视为 $\vec{A}_k$ 的特殊情况,具体原因可见于下文。
\end{itemize}

\end{document}
